%Carattere dimensione 12
\documentclass[12pt]{article}

%Margini e interlinea
\usepackage[top=1in, bottom=1in, left=1.2in, right=1in]{geometry}
\pagestyle{plain}
\linespread{1.5}

%Librerie utili
\usepackage[english]{babel}
\usepackage[utf8]{inputenc}
\usepackage{libertine}
\usepackage{graphicx}
\usepackage{floatflt}
\usepackage{blindtext}
\usepackage{enumitem}
\usepackage{amsthm}
\usepackage{subfig}
\usepackage{listings}
\usepackage{listingsutf8}
\usepackage{amsmath}
\usepackage{framed}
\usepackage{minibox}
\usepackage{float}
\usepackage{wrapfig}
\usepackage{longtable}
\usepackage[strict]{changepage}
\usepackage{pgfplots}
\usepackage{tikz}
\usetikzlibrary{matrix}
\pgfplotsset{width=11cm,compat=1.9}
\usepgfplotslibrary{external}
\tikzexternalize

\begin{document}

\begin{titlepage}
\begin{figure}[t]
	\centering\includegraphics[width=0.9\textwidth]{media/scritta}
    \centering\includegraphics[width=0.4\textwidth]{media/logo}
\end{figure}

\begin{center}
	\textbf{ Department of Computer Science\\ Degree course in Computer Science\\}
	\vspace{15mm}
    {\LARGE{\bf Machine learning techniques}}\\
	\vspace{3mm}
	{\LARGE{\bf for 3D point clouds}}\\
\end{center}

\vspace{20mm}

\begin{minipage}[t]{0.47\textwidth}
	{\large{\bf Relatori:\\ Prof. Davide Bacciu\\ Prof. Daniela Giorgi \\ Prof. Paolo Cignioni \\ Prof. Francesco Banterle}}
\end{minipage}\hfill\begin{minipage}[t]{0.47\textwidth}\raggedleft
	{\large{\bf Presented by: \\ Mattia Angelini\\ }}
\end{minipage}

\vspace{18mm}

\centering{\large{\bf Winter Session\\ Academic Year 2018/2019 }}

\end{titlepage}

%Table of contents ---------------------------------
\tableofcontents
\clearpage
%---------------------------------------------------

\section{Introduction}
3D data can be represented in many different ways, 
where each one inspired new approaches to Deep Learning techniques with their pro and cons.
\\
However in the last years, the importance of 3D point clouds have risen consistently, 
due to the use in the sensor of smartphones and self-driving cars, 
but despite that, the focus remains only on how to process those data improving on the models themselves, 
without put efforts in how to encode those data.
\\
We then have chosen to direct our efforts, 
on how to improve on the encoding of the information on 3D Point Clouds and, 
see how this reflects on the results of the currently developed models.

\subsection{Other 3D techniques}
Before proceeding, we would like to introduce other machine learning techniques on 3D data, 
starting with the \textbf{view-based} approach.
\\
\textbf{View-based} systems use a group of view of a 3D object to capture its spatial information[]\, 
enabling us to use the success achieved in the Deep Learning for images like the CNN[] directly to the views.
This, however, means that we need many views and usually is very memory intensive even if the average speed is good.
\\
\textbf{Voxels}, on the other hand, 
encode the information of the 3D data in a regular 3-dimensional grid, where each cube represent a unit of volume in which the 3D is stored, 
giving us a volumetric representation.
Having a 3D grid means that we could extend the techniques like the convolution to this data rapresentation[ref to voxnet], 
at the cost of computational complexity passing by $ O(n^2) $ to $ O(n^3) $, new techniques improve on this using a subdivision of the grid in OctTrees with patches to aggregate groups of Voxels[A-oCNN].
\\
In the end, \textbf{Surface-based} representation, 
aims to understand the information of objects using their superficial information, 
however, working with this type of data is not straightforward and usually what is done is to encode the mesh in a 2-dimensional space[], 
at this point, we can use the well-asserted techniques for images. 
But they can create some artefacts and require a good mesh to work with.  

\subsection{Learning on graphs}
We have decide to introduce also the results achived in the deep learning applied to graphs, 
and see the proble also from this point of view, and see what the results achived in this field can done for this type of data.


%---------------------------------------------------
% Start of the Background about the point cloud data
%---------------------------------------------------
\section{Background and Related Works}

\subsection{3D Point Clouds}

\subsubsection{Pointnet}




%---------------------------------------------------
% Start part about what we have tried
%---------------------------------------------------
\section{Techniques for 3D Point Clouds}

\subsection{Sampling}

\

%---------------------------------------------------
% Start part about what we have tried
%---------------------------------------------------
\section{Results}


\section{Observations}


\section{Conclusions}

\appendix

\bibliography{references}
\bibliographystyle{plain}


\end{document}